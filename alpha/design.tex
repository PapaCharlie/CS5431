\documentclass{article}
\usepackage[margin=2cm, headheight=0pt, headsep=1cm, includeheadfoot, top=0.75cm, bottom=1cm]{geometry}
\usepackage{enumerate, fancyhdr, graphicx, amsmath, float, url, hyperref, color}
\usepackage{array,booktabs,dirtree,subcaption}

\title{Vault 5431 - Design}
\author{Alicia Wu, Britney Wong, Chang Yang Jiao, Paul Chesnais}
\date{\today}

\pagestyle{fancy}
\fancyhead{}
\lhead{cj285, bmw227, pmc85, yw344}
\chead{Vault 5431 - Sprint Report}
\rhead{\today}
\fancyfoot{}
\rfoot{\thepage}
\lfoot{\includegraphics[height=20pt]{figures/Logo}}
\renewcommand{\headrulewidth}{0.5pt}
\renewcommand{\footrulewidth}{0.5pt}

\begin{document}
\maketitle
\thispagestyle{empty}
\section{Filesystem Design}

\begin{figure}[H]
  \centering
  \begin{subfigure}[b]{0.3\textwidth}
    \dirtree{%
    .1 .vault5431/.
    .2 (username hash).
    .3 crypto.key.
    .3 id\_rsa.crypto.
    .3 id\_rsa.crypto.pub.
    .3 id\_rsa.signing.
    .3 id\_rsa.signing.pub.
    .3 iv.crypto.
    .3 iv.signing.
    .3 log.
    .3 password.hash.
    .3 signing.key.
    .3 vault.
    .2 log.
    }
    \caption{Current Structure}
  \end{subfigure}~
  \begin{subfigure}[b]{0.3\textwidth}
    \dirtree{%
    .1 .vault5431/.
    .2 (username hash).
    .3 id\_rsa.crypto.
    .3 id\_rsa.signing.
    .3 log.
    .3 password.hash.
    .3 vault.
    .2 log.
    .2 pubkey\_dict.
    .2 pubkey\_dict.sig.
    }
    \caption{Planned Structure}
  \end{subfigure}
  \caption{Directory Structure}
  \label{fig:directory_structure}
\end{figure}

\subsection{Home Directories}
\par All of the relevant data is stored in a directory called \texttt{.vault5431} in server's root. All of a given user's information and data is stored in their respective sub-directories.
\paragraph{Current structure} As of right now, each user is assigned a pair of encryption keys, a pair of signing keys, a symmetric encryption key and a symmetric signing key at user creation. Both private keys are encrypted under the user's master password. The symmetric keys are encrypted under the encryption private key. A hashed version of the password is created using PBKDF2 and stored in under \texttt{password.hash}. An empty log file is created and encrypted under the user's public key. An empty vault file is created and encrypted under the symmetric cryptographic key. Though the encryption is currently turned off, the system is set up in such a way that when it wants to append to a given user's log, it encrypts under their private key, and simply appends it to the corresponding \texttt{log} file. Similarly, to decrypt the vault, the user must provide the master password to decrypt the private encryption key, which will first be checked against the hashed version, then used to decrypt the symmetric key and finally decrypt the vault if it is correct. The point of this nested encryption was to be able to avoid having to reencrypt the entire vault upon master password change. Though the signing keys are not currently in use, they will serve to sign passwords when/if we implement password sharing.
\paragraph{Planned structure} Much of the current structure will be kept, but a lot of files will be removed from the user's directory. In the case that an attacker has compromised the machine and is able to edit the filesystem, there is currently nothing in place to prevent them from swapping out a user's public key and gaining access to all further log entries. Additionally, it was decided that having to reencrypt the whole Vault every time the master password was easier than having the nested encryption mentioned above. Additionally, it does not make cracking the Vault so much harder for the attacker than the lost time was made up. Therefore the symmetric keys \texttt{crypto.key} and \texttt{signing.key} will be phased out in the next iteration, and each public key will stored in a signed and trusted ``phonebook'' of sorts (see \texttt{pubkey\_dict} and \texttt{pubkey\_dict.sig}), out of the reach of attackers. All of this will alleviate runtime costs, reduce potential errors and increase the Integrity of the logs, at no measurable cost to the Integrity and Confidentiality of the Vault. The master admin signing and encryption keys will be derived from a master password that only the fours of us know. Even if this password were to be leaked, the user's logs and Vaults will not be at risk.

\subsection{Rationale}
\par We decided to use a filesystem design over a database for two main reasons. First, it immediately prevents SQL injections from occurring, which is one less thing our system needs to worry about. Second, it makes it simpler to manage encryption of and signatures for individual files, which we will be extensively using to ensure the integrity and confidentiality of the user data.

\section{Audit}
For the audit section of our project, we organized our logs based on two types- the user logs and the system log. Currently, they are formatted very similarly but we made the distinction in case their uses begin to deviate in later sprints.
\subsection{User Logs}
\par All the information in our system is organized in a file system structure. Therefore, each user has his own directory containing his log files (base64 encoded), password files, keys, and other information found in his vault (possibly secure notes time permitting). Refer to section 1 for this directory layout. Each user log entry has six fields - log type (debug, info, warning, or error), the IP address that completed the specified action, the hashed username, the timestamp of when the action was initiated, signature, and a message that specifies the type of action that took place. The Vault username is hashed to prevent a user's username from being discovered in the case where an attacker somehow manages to get the log. We believe that because it is reasonable for a user to use same username in multiple places (to reduce memorization burden), an attacker should not be able to discover a username even if he successfully attacks our system. The signature is a system signature that is produced to indicate that the log entry was written by the system rather than some outside attacker.\\

\par Currently, a log entry is created every time a new password is created or changed, or when the log is accessed. Later, log entries will be created when a password is deleted, when there is an unsuccessful login attempt to a given username, and when settings are changed. These do not yet exist because their corresponding functionalities were not implemented in this sprint.  The log entries are currently stored in CSV format as base64 encoded strings. We plan to encrypt them in later sprints.\\

\par Finally, a user can currently see his personal log by successfully logging into his account and clicking on the log link on their dashboard. From there, he will be able to see each log entry with four fields – the log type, the IP address, timestamp, and message with the action. The signature validation, when it is implemented, will be done behind the scenes before the log is displayed.

\subsection{System Log}
The system log should only be viewed by the admins of our system (the four of us). We will have our own system keys to be able to access the system log (currently base64 encoded), and the system log entry fields are the same as for the user log. We create log entries for actions done by the server, as well as log entries for error events (ex: unsuccessful login events, user creation, etc.), so we have the information to recognize a potential attack on our system. Later on, we hope to use this log to help us implement the functionality of freezing a user vault when too many unsuccessful login attempts are made.

\section{Authentication}
\section{Authorization}
\section{Confidentiality}
\section{Integrity}

\end{document}
