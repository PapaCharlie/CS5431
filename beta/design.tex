\documentclass{article}
\usepackage[margin=2cm, headheight=0pt, headsep=1cm, includeheadfoot, top=0.75cm, bottom=1cm]{geometry}
\usepackage{enumerate, fancyhdr, graphicx, amsmath, float, url, hyperref, color}
\usepackage{array,booktabs,dirtree,subcaption,nameref}

\title{Vault 5431 - Design}
\author{Alicia Wu, Britney Wong, Chang Yang Jiao, Paul Chesnais}
\date{\today}

\pagestyle{fancy}
\fancyhead{}
\lhead{cj285, bmw227, pmc85, yw344}
\chead{Vault 5431 - Sprint Report}
\rhead{\today}
\fancyfoot{}
\rfoot{\thepage}
\lfoot{\includegraphics[height=20pt]{figures/Logo}}
\renewcommand{\headrulewidth}{0.5pt}
\renewcommand{\footrulewidth}{0.5pt}

\begin{document}
\maketitle
\thispagestyle{empty}

\section{Server Environment}
\label{sec:server_environment}

\subsection{Platform}
\label{sub:platform}
\par The server is running Ubuntu 15.10 with weekly backups. SSH access is only allowed for SysAdmins whose keys have been cleared, with password login disabled.

\subsection{Server Startup}
\label{sub:server_startup}
\par The first thing to do is to start the VaultRedirector server, which simply listens to all HTTP requests coming on port 80 and redirects to ``/'' on port 443 (dropping all of the request's contents in the process). This is to force users to use HTTPS and prevent them from accidentally sending information via an insecure channel.
\par Next, the Vault server needs to be started up. The program requires access to three files: the \texttt{/root/.vault5431/} directory that stores all of the required data (see \nameref{sub:filesystem_design}), the Java KeyStore \texttt{keystore.jks} that contains the encrypted private key for the SSL certificate and the KeyStore \texttt{truststore.jks} that contains manually trusted certificates for the SMS messaging service. Upon startup, the program prompts the SysAdmin for three passwords: the admin password, from which the Admin Signing Key (ASK) and Admin Encryption Key (AEK) are generated using PBKDF2, the password under which the SSL certificate's private key is encrypted and the password used to verify the trust store's authenticity. The Admin password is only known to the four members of the team.

\subsection{Filesystem Design}
\label{sub:filesystem_design}

\begin{figure}[H]
  \centering
  \begin{subfigure}[b]{0.3\textwidth}
    \dirtree{%
      .1 .vault5431/.
      .2 (username hash).
      .3 id\_rsa.crypto.
      .3 id\_rsa.crypto.pub.
      .3 id\_rsa.crypto.pub.sig.
      .3 id\_rsa.signing.
      .3 id\_rsa.signing.pub.
      .3 id\_rsa.signing.pub.sig.
      .3 log.
      .3 password.hash.
      .3 phone.number.
      .3 vault.
      .3 vault.salt.
      .2 log.
      .2 admin.salt.
    }
  \end{subfigure}
  \caption{Directory Structure}
  \label{fig:directory_structure}
\end{figure}

\par All of the relevant data is stored in a directory called \texttt{.vault5431} in server's root. All of a given user's information and data is stored in their respective sub-directories. The system log is stored in \texttt{.vault5431/log}, and the salt used to derive the AEK and ASK is stored in \texttt{.vault5431/admin.salt}.
\par At creation, each user is assigned their home directory, named after the hash of their username. Two pairs of 4096 bit RSA keys are generated: one for encryption and one for signing. Both private keys are encrypted under the AEK, and the public keys are signed with the ASK, so that if an attacker gains access to disk, they cannot swap out the public key for another for which they know the private key, and view a user's logs. The private key encryption will be changed to happen client side so that only the user can decrypt them. A hashed version of the password is created using PBKDF2 and stored in under \texttt{password.hash}. An empty log file is created and encrypted under the user's public key. An empty vault file is created, to contain the encrypted data given by the user. Additionally, a random salt is generated and encrypted under the AEK. This salt will be used to derive the final, client side encryption key. This is to make decrypting a user's vault required both the system's authorization and the client's master password. Their phone number is encrypted under the AEK and stored in \texttt{phone.number}.

\subsection{Rationale}
\par We decided to use a filesystem design over a database for two main reasons. First, it immediately prevents SQL injections from occurring, which is one less thing our system needs to worry about. Second, it makes it simpler to manage encryption of and signatures for individual files, which we will be extensively using to ensure the integrity and confidentiality of the user data.

\section{Audit}
For the audit section of our project, we organized our logs based on two types- the user logs and the system log. Currently, they are formatted very similarly but we made the distinction in case their uses begin to deviate in later sprints.
\subsection{User Logs}
\par All the information in our system is organized in a file system structure. Therefore, each user has his own directory containing his log files (base64 encoded), password files, keys, and other information found in his vault (possibly secure notes time permitting). Refer to section 1 for this directory layout. Each user log entry has six fields - log type (debug, info, warning, or error), the IP address that completed the specified action, the hashed username, the timestamp of when the action was initiated, signature, and a message that specifies the type of action that took place. The Vault username is hashed to prevent a user's username from being discovered in the case where an attacker somehow manages to get the log. We believe that because it is reasonable for a user to use same username in multiple places (to reduce memorization burden), an attacker should not be able to discover a username even if he successfully attacks our system. The signature is a system signature that is produced to indicate that the log entry was written by the system rather than some outside attacker.

\par Currently, a log entry is created every time a new password is created, changed or deleted and when the log is accessed, additionally successful or failed logins are also logged.

\par Finally, a user can currently see his personal log by successfully logging into his account and clicking on the log link on their dashboard. From there, he will be able to see each log entry with four fields – the log type, the IP address, timestamp, and message with the action. The signature validation, when it is implemented, will be done behind the scenes before the log is displayed.

\subsection{System Log}
The system log should only be viewed by the admins of our system (the four of us). We will have our own system keys to be able to access the system log (currently base64 encoded), and the system log entry fields are the same as for the user log. We create log entries for actions done by the server, as well as log entries for error events (ex: unsuccessful login events, user creation, etc.), so we have the information to recognize a potential attack on our system. Later on, we hope to use this log to help us implement the functionality of freezing a user vault when too many unsuccessful login attempts are made.

\section{Authentication}

\subsection{User Creation}
\par A user is created when they submit a username and password combination as well as their phone number. If the username is taken, the user will be notified. We found that there was no way to completely prevent an attacker from knowing if a username is a valid one in our system or not because they will be notified on sign up. The client does not send the password in plaintext, rather it sends the result of hashing it twice. This is to prevent the server from seeing the plaintext password, but also because the first hash will actually be used to derive the master key that decrypts the vault.

\subsection{Client-Side Authentication}
\par The authentication happens in two steps. First, the user is required to type in their username and password. Afterwards, an authentication code will be sent to the user's cellphone as an SMS and the input of this code will be required to access the user vault. Below, each step is explained in greater detail.

\subsubsection{Username and Password}
\par The client sends the username and the second hash of the password to the server, while storing the first hash in the browser's \texttt{sessionStorage}. The server runs PBKDF2 on the second hash to verify that it is indeed the correct password. If so, the user is assigned an unverified token (see below). Otherwise, the user is redirected to the login page. The reason the first hash of the master password in the session storage instead of the plaintext is because the session storage is readily available for all to see simply by checking the resources in the web browser. Quite obviously, we do not want the master password to be publicly displayed. We ensured that no information about the username and password combination is revealed when an incorrect username or password is inputted. All the user will see is that no such combination of username and password is valid.

\subsubsection{Two Factor Authentication (2FA)}
\par After the correct password/username combination is inputted, the server will send the user a 6 digit code with a 3 minute expiration time. The point of this code is to authenticate an user based on something that they have (the code, and thus their SMS-receiving phone) in addition to something that they know. This ensures that even if an attacker somehow cracked the user master password, they still would not have access to the vault. We assume that the user's phone is safely within his/her hands. We plan to limit the rate in which a person can enter their log in code. If the user correctly enters the 6 digit code, their token is upgraded to a verified token (see below), and they are able to view their vault.

\subsubsection{Decrypting the Vault}
After the user successfully completes 2FA and has received a verified token, the system sends the client the encrypted vault along with the decrypted vault salt. Now, the client hashes the vault salt and the first hash of the master password to form the final master key. This key will be used to encrypt and decrypt the passwords. As can be seen, because the hashing function (AES-128 with CCM) is considered secure and is one way, only the user can ever decrypt the vault. Furthermore, this ensures that even if an attacker somehow gains access to the server or its backups, they still cannot decrypt the password vault because the user key is never stored.

\subsection{Tokens}
\label{sub:tokens}
\par Once a user has been authenticated (see below), it would be ideal if they could remain authenticated for the remainder of their session. This is the point of Tokens. Simply put, Tokens are a piece of unforgeable evidence that convinces the server that the user has been authenticated, meaning that it is okay to send the encrypted Vault to that user.

\subsubsection{Secure Cookies}
\label{ssub:secure_cookies}
\par Most modern browsers offer a notion of ``Secure Cookies''. Secure cookies are only sent to the appropriate server if the connection to said server is secured over HTTPS, meaning that it is presumably safe to store the tokens in secure cookies. Cookies persist through multiple connections, and are a useful way to maintain state across multiple requests to the server within a session.

\subsection{Token Generation and Signing}
\label{sub:token_generation_and_signing}
\par The tokens used in this system were inspired by JSON Web Tokens (see RFC 7519\cite{bib:RFC7519}), but modified to better suit the system. Here are the fields contained in the token:
\begin{description}
  \item[\texttt{username}] User id
  \item[\texttt{creationDate}] Creation time
  \item[\texttt{expiresAt}] Expiration time
  \item[\texttt{id}] Token id
  \item[\texttt{verified}] Boolean indicating whether or not the user has gone through 2FA.
  \item[\texttt{signature}] Token signature.
\end{description}
\par The first two fields are rather self explanatory. If a token is presented to the server, but the current time is not between its creation time or its expiration time, it is immediately rejected. The user id represents which user this token was generated for. The token id is a random UUID assigned at token creation, and will be used to do token revocation, and keeping track of how many tokens have been assigned per day per user. The boolean field is used to indicate that the user has indeed put in the correct password, but has not been 2 factor authenticated. This is to prevent anyone but the user him/herself to attempt 2FA, but also prevents partially authenticated users from actually viewing the encrypted vault. Finally, all of the above fields are signed under the current Rolling Key.
\par Token creation is very simple: when the user puts in the correct password, a token is generated with the \texttt{verified} field set to false, then signed. When the user successfully passes 2FA, the token is verified, singed again, and sent back.

\subsubsection{Rolling Keys}
\label{ssub:rolling_keys}
\par Every night, at midnight EST, the server changes signing keys. This automatically voids any tokens assigned the day before, and reduces the window in which an attacker may crack the signing key. The current Rolling Key is always generated from a SecureRandom (Java's cryptographically strong random number generator) instance, making it impossible to guess the current key based on previous ones, were they ever to be cracked.

\subsection{Token validation}
\label{sub:token_validation}
\par Here is the steps the system takes to validate tokens:
\begin{enumerate}[1.]
  \item The first thing to do is check whether or not the current time is between the token's purported creation time and expiration time. If not, the token is rejected, and the user must sign in again.
  \item If the current time is within the token's validity window, then the signature is verified against the whole token. If the signature matches, then the previous time check was executed on valid dates, and the process can continue. If the signature does not match, indicating someone may have tampered with it, the token is rejected. Incidentally, the signature of a token from a previous day will never match because the keys used to sign it has changed, voiding it by default. This behavior is desired because an old valid token should not be a reason to assign a new valid token, otherwise if an attacker gains hold of a token, they will be authenticated forever.
  \item If the signature matches, then the token's id is checked to see whether or not that token was voided through other means, i.e. by user action. If the token is void, it is rejected.
  \item Finally, the presented token is checked against the user's request. If the token is verified, the user will be allowed to see their vault and logs. Otherwise, the user will only be allowed to view the 2FA page. On the other hand, only a user with a valid, but unverified token can view the 2FA page, to ensure that only the desired user may attempt 2FA.
\end{enumerate}

\section{Authorization}
\section{Confidentiality}
\section{Integrity}

\begin{thebibliography}{30}
  \bibitem{bib:RFC7519}
    \url{https://www.ietf.org/rfc/rfc7519.txt}
\end{thebibliography}

\end{document}
